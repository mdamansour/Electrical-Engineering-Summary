\documentclass[12pt]{article}

% Setting up the page geometry
\usepackage[margin=1in]{geometry}

% Including essential packages for mathematical typesetting
\usepackage{amsmath, amssymb, amsfonts}
\usepackage{siunitx} % For units and numerical formatting
\usepackage{enumitem} % For customized lists
\usepackage{booktabs} % For professional tables
\usepackage{xcolor} % For colored text
\usepackage{tocloft} % For table of contents customization

% Configuring the table of contents
\renewcommand{\cftsecleader}{\cftdotfill{\cftdotsep}}

% Setting up fonts
\usepackage{times} % Using Times for better readability
\usepackage[T1]{fontenc}

% Customizing section headers
\usepackage{sectsty}
\sectionfont{\large\bfseries}
\subsectionfont{\normalsize\bfseries}

% Defining custom commands for consistent formatting
\newcommand{\concept}[1]{\textbf{#1}}
\newcommand{\formula}[1]{\textit{Formula: }#1}

\begin{document}

% Creating the title page
\title{Electrical Engineering Preparation Summary}
\author{Your Name}
\date{August 25, 2025}
\maketitle

% Adding table of contents
\tableofcontents
\newpage

% Starting the main content
\section{Introduction}
This document provides a concise summary of key electrical engineering concepts focused on electrical machines, including electromotive force (EMF), radial speed, and current flow, to aid in preparation.

\section{Electrical Machines}
\subsection{Electromotive Force (EMF)}
\concept{Electromotive Force (EMF)} is the voltage generated in a conductor moving through a magnetic field, such as in a generator or motor.
\[
\formula{E = \frac{p N \Phi n}{a}}
\]
\begin{itemize}
    \item \textbf{Parameters}:
        \begin{itemize}
            \item \(E\): Induced EMF (\si{\volt}).
            \item \(p\): Number of poles in the machine.
            \item \(N\): Number of conductors.
            \item \(\Phi\): Magnetic flux per pole (\si{\weber}).
            \item \(n\): Rotational speed (\si{\radian\per\second}).
            \item \(a\): Number of parallel paths in the winding.
        \end{itemize}
    \item \textbf{Explanation}: The EMF is proportional to the number of poles, conductors, magnetic flux, and rotational speed, divided by the parallel paths. This is critical for understanding voltage generation in machines like generators.
    \item \textbf{Example}: A 4-pole generator with 100 conductors, a flux of \SI{0.02}{\weber}, speed of \SI{1200}{rpm}, and 2 parallel paths generates an EMF calculated by converting speed to radians per second (see below) and applying the formula.
\end{itemize}

\subsection{Radial Speed}
\concept{Radial Speed} (or angular speed, \(\omega\)) represents how fast a rotor spins in a rotating machine, measured in \si{\radian\per\second}.
\[
\formula{\omega = \frac{2 \pi n}{60}}
\]
\begin{itemize}
    \item \textbf{Parameters}:
        \begin{itemize}
            \item \(\omega\): Angular speed (\si{\radian\per\second}).
            \item \(n\): Rotational speed in revolutions per minute (RPM).
        \end{itemize}
    \item \textbf{Simple Explanation}: This formula converts rotational speed from RPM to radians per second. One revolution equals \(2\pi\) radians, and dividing by 60 adjusts for seconds in a minute. For example, at 1200 RPM:
        \[
        \omega = \frac{2 \pi \cdot 1200}{60} = 40\pi \approx \SI{125.66}{\radian\per\second}.
        \]
    \item \textbf{Why It Matters}: Angular speed is used in EMF calculations to determine how fast the magnetic field changes, directly affecting voltage generation.
\end{itemize}

\subsection{Current Flow in the Circuit}
\concept{Current Flow} in a circuit (induit) of an electrical machine is determined by the difference between the EMF and the voltage across the circuit, accounting for the internal resistance.
\[
\formula{I = \frac{E - V}{R}}
\]
\begin{itemize}
    \item \textbf{Parameters}:
        \begin{itemize}
            \item \(I\): Current flowing through the circuit (\si{\ampere}).
            \item \(E\): Electromotive force (EMF, \si{\volt}).
            \item \(V\): Voltage across the circuit (\si{\volt}).
            \item \(R\): Internal resistance of the circuit (\si{\ohm}).
        \end{itemize}
    \item \textbf{Explanation}: The current is driven by the difference between the generated EMF and the circuit’s terminal voltage, divided by the internal resistance. The power consumed by the internal resistor is \(P = I^2 R\), where \(I\) results from the voltage drop across the resistor, controlling current flow.
    \item \textbf{Example}: For an EMF of \SI{100}{\volt}, a circuit voltage of \SI{90}{\volt}, and an internal resistance of \SI{2}{\ohm}, the current is:
        \[
        I = \frac{100 - 90}{2} = \SI{5}{\ampere}.
        \]
        The power consumed by the internal resistor is \(P = 5^2 \cdot 2 = \SI{50}{\watt}\).
\end{itemize}

\end{document}