\documentclass[12pt]{article}

% Setting up the page geometry
\usepackage[margin=1in]{geometry}

% Including essential packages for mathematical typesetting
\usepackage{amsmath, amssymb, amsfonts}
\usepackage{siunitx} % For units and numerical formatting
\usepackage{enumitem} % For customized lists
\usepackage{booktabs} % For professional tables
\usepackage{xcolor} % For colored text
\usepackage{tocloft} % For table of contents customization

% Configuring the table of contents
\renewcommand{\cftsecleader}{\cftdotfill{\cftdotsep}}

% Setting up fonts
\usepackage{times} % Using Times for better readability
\usepackage[T1]{fontenc}

% Customizing section headers
\usepackage{sectsty}
\sectionfont{\large\bfseries}
\subsectionfont{\normalsize\bfseries}

% Defining custom commands for consistent formatting
\newcommand{\concept}[1]{\textbf{#1}}
\newcommand{\formula}[1]{\textit{Formula: }#1}

\begin{document}

% Creating the title page
\title{Electrical Engineering Preparation Summary}
\author{Your Name}
\date{August 25, 2025}
\maketitle

% Adding table of contents
\tableofcontents
\newpage

% Starting the main content
\section{Introduction}
This document provides a concise summary of key electrical engineering concepts focused on electrical machines and RLC circuits. It covers electromotive force (EMF), radial speed, current flow, electromagnetic torque, and the resonance curve in series RLC circuits to aid in preparation.

\section{Electrical Machines}
\subsection{Electromotive Force (EMF)}
\concept{Electromotive Force (EMF)} is the voltage generated in a conductor moving through a magnetic field, such as in a generator or motor.
\[
\formula{E = \frac{p N \Phi n}{a}}
\]
\begin{itemize}
    \item \textbf{Parameters}:
        \begin{itemize}
            \item \(E\): Induced EMF (\si{\volt}).
            \item \(p\): Number of poles in the machine.
            \item \(N\): Number of conductors.
            \item \(\Phi\): Magnetic flux per pole (\si{\weber}).
            \item \(n\): Rotational speed (\si{\radian\per\second}).
            \item \(a\): Number of parallel paths in the winding.
        \end{itemize}
    \item \textbf{Explanation}: The EMF is proportional to the number of poles, conductors, magnetic flux, and rotational speed, divided by the parallel paths. This is critical for understanding voltage generation in machines like generators.
    \item \textbf{Example}: A 4-pole generator with 100 conductors, a flux of \SI{0.02}{\weber}, speed of \SI{1200}{rpm}, and 2 parallel paths generates an EMF calculated by converting speed to radians per second (see below) and applying the formula.
\end{itemize}

\subsection{Radial Speed}
\concept{Radial Speed} (or angular speed, \(\omega\)) represents how fast a rotor spins in a rotating machine, measured in \si{\radian\per\second}.
\[
\formula{\omega = \frac{2 \pi n}{60}}
\]
\begin{itemize}
    \item \textbf{Parameters}:
        \begin{itemize}
            \item \(\omega\): Angular speed (\si{\radian\per\second}).
            \item \(n\): Rotational speed in revolutions per minute (RPM).
        \end{itemize}
    \item \textbf{Simple Explanation}: This formula converts rotational speed from RPM to radians per second. One revolution equals \(2\pi\) radians, and dividing by 60 adjusts for seconds in a minute. For example, at 1200 RPM:
        \[
        \omega = \frac{2 \pi \cdot 1200}{60} = 40\pi \approx \SI{125.66}{\radian\per\second}.
        \]
    \item \textbf{Why It Matters}: Angular speed is used in EMF and torque calculations to determine how fast the magnetic field changes, affecting voltage and mechanical output.
\end{itemize}

\subsection{Current Flow in the Circuit}
\concept{Current Flow} in a circuit (armature) of an electrical machine is determined by the difference between the EMF and the voltage across the circuit, accounting for the internal resistance.
\[
\formula{I = \frac{E - V}{R}}
\]
\begin{itemize}
    \item \textbf{Parameters}:
        \begin{itemize}
            \item \(I\): Current flowing through the circuit (\si{\ampere}).
            \item \(E\): Electromotive force (EMF, \si{\volt}).
            \item \(V\): Voltage across the circuit (\si{\volt}).
            \item \(R\): Internal resistance of the circuit (\si{\ohm}).
        \end{itemize}
    \item \textbf{Explanation}: The current is driven by the difference between the generated EMF and the circuit’s terminal voltage, divided by the internal resistance. The power consumed by the internal resistor is \(P = I^2 R\), where \(I\) results from the voltage drop across the resistor, controlling current flow.
    \item \textbf{Example}: For an EMF of \SI{100}{\volt}, a circuit voltage of \SI{90}{\volt}, and an internal resistance of \SI{2}{\ohm}, the current is:
        \[
        I = \frac{100 - 90}{2} = \SI{5}{\ampere}.
        \]
        The power consumed by the internal resistor is \(P = 5^2 \cdot 2 = \SI{50}{\watt}\).
\end{itemize}

\subsection{Electromagnetic Torque}
\concept{Electromagnetic Torque} (\(C\)) is the rotational force produced in an electrical machine, such as a motor, due to the interaction of current and magnetic flux.
\[
\formula{C = \frac{E \cdot I}{\omega}}
\]
\begin{itemize}
    \item \textbf{Parameters}:
        \begin{itemize}
            \item \(C\): Torque (\si{\newton\metre}).
            \item \(E\): Induced EMF (\si{\volt}).
            \item \(I\): Current in the circuit (\si{\ampere}).
            \item \(\omega\): Angular speed (\si{\radian\per\second}).
        \end{itemize}
    \item \textbf{Alternative Formula}:
        \[
        \formula{C = \frac{p}{a} \cdot \frac{N \cdot \Phi \cdot I}{2 \pi}}
        \]
        \begin{itemize}
            \item \(p\): Number of poles.
            \item \(a\): Number of parallel paths.
            \item \(N\): Number of conductors.
            \item \(\Phi\): Magnetic flux per pole (\si{\weber}).
            \item \(I\): Current (\si{\ampere}).
        \end{itemize}
    \item \textbf{Explanation}: Torque is the mechanical output of an electrical machine, resulting from the interaction of the EMF and current, divided by angular speed in the first formula. The alternative formula expresses torque in terms of the machine’s physical parameters, showing how the number of poles, conductors, flux, and current contribute to the rotational force. Both yield torque in \si{\newton\metre}.
    \item \textbf{Example}: For a machine with \(E = \SI{100}{\volt}\), \(I = \SI{5}{\ampere}\), and \(\omega = \SI{125.66}{\radian\per\second}\) (from 1200 RPM), the torque is:
        \[
        C = \frac{100 \cdot 5}{125.66} \approx \SI{3.98}{\newton\metre}.
        \]
        Using the alternative formula, for \(p = 4\), \(a = 2\), \(N = 100\), \(\Phi = \SI{0.02}{\weber}\), and \(I = \SI{5}{\ampere}\):
        \[
        C = \frac{4}{2} \cdot \frac{100 \cdot 0.02 \cdot 5}{2 \pi} \approx \SI{3.18}{\newton\metre}.
        \]
        (Note: Discrepancies may arise due to simplifying assumptions; verify with specific machine constants.)
\end{itemize}

\subsection{Determining Motor or Generator Operation}
\concept{Motor vs. Generator Operation} can be determined by comparing the electromotive force (EMF) and the voltage across the armature (\(V\)) of the electrical machine.
\begin{itemize}
    \item \textbf{Explanation}: 
        \begin{itemize}
            \item \textbf{Generator Mode}: If the EMF (\(E\)) is greater than the armature voltage (\(V\)), i.e., \(E > V\), the machine is operating as a generator. This is because the machine generates a higher voltage than the terminal voltage, driving current outward to supply power to an external load.
            \item \textbf{Motor Mode}: If the EMF (\(E\)) is less than the armature voltage (\(V\)), i.e., \(E < V\), the machine is operating as a motor. In this case, an external voltage source supplies a higher voltage, driving current through the armature to produce mechanical work.
            \item The current flow formula \(I = \frac{E - V}{R}\) determines the direction and magnitude of the current. In generator mode (\(E > V\)), current flows out of the machine (positive \(I\)). In motor mode (\(E < V\)), current flows into the machine (negative \(I\)).
        \end{itemize}
    \item \textbf{Example}: For a machine with an internal resistance \(R = \SI{2}{\ohm}\):
        \begin{itemize}
            \item If \(E = \SI{100}{\volt}\) and \(V = \SI{90}{\volt}\), then \(E > V\), and \(I = \frac{100 - 90}{2} = \SI{5}{\ampere}\). The positive current indicates generator operation, supplying power.
            \item If \(E = \SI{90}{\volt}\) and \(V = \SI{100}{\volt}\), then \(E < V\), and \(I = \frac{90 - 100}{2} = \SI{-5}{\ampere}\). The negative current indicates motor operation, consuming power to produce mechanical work.
        \end{itemize}
    \item \textbf{Why It Matters}: Comparing \(E\) and \(V\) allows quick identification of the machine’s operating mode, critical for analyzing performance and efficiency in electrical systems.
\end{itemize}

\section{RLC Circuits}
\subsection{Resonance Curve (\textit{la courbe de résonance})}
\concept{Resonance Curve} (\textit{la courbe de résonance}) is a graph showing the variation of current intensity (or voltage across a circuit element) as a function of frequency (\(f\)) or angular frequency (\(\omega\)) in a series RLC circuit.
\begin{itemize}
    \item \textbf{Explanation}: In a series RLC circuit, consisting of a resistor (\(R\)), inductor (\(L\)), and capacitor (\(C\)), the resonance curve illustrates how the current or voltage across a component (e.g., resistor, inductor, or capacitor) changes with the input signal’s frequency. At the resonance frequency, the inductive and capacitive reactances cancel each other out, resulting in maximum current (or minimum impedance) in the circuit. The resonance frequency occurs when:
        \[
        \omega = \frac{1}{\sqrt{L C}},
        \]
        where \(\omega\) is the angular frequency (\si{\radian\per\second}), \(L\) is inductance (\si{\henry}), and \(C\) is capacitance (\si{\farad}). The curve peaks at this frequency, showing a sharp increase in current, and decreases on either side, forming a bell-shaped graph.
    \item \textbf{Key Characteristics}:
        \begin{itemize}
            \item The peak current occurs at the resonance frequency, where the circuit’s impedance is purely resistive (\(Z = R\)).
            \item The sharpness of the curve depends on the quality factor (\(Q\)), which is higher for lower resistance, indicating a narrower peak.
            \item The resonance curve is critical for applications like tuning circuits in radios or filters in signal processing.
        \end{itemize}
    \item \textbf{Example}: In a series RLC circuit with \(R = \SI{10}{\ohm}\), \(L = \SI{0.1}{\henry}\), and \(C = \SI{100}{\micro\farad}\), the resonance angular frequency is:
        \[
        \omega = \frac{1}{\sqrt{0.1 \cdot 100 \cdot 10^{-6}}} \approx \SI{316.23}{\radian\per\second},
        \]
        corresponding to a frequency of \(f = \frac{\omega}{2 \pi} \approx \SI{50.33}{\hertz}\). Plotting the current versus frequency would show a peak at this value.
    \item \textbf{Why It Matters}: The resonance curve helps engineers analyze and design circuits for specific frequencies, optimizing performance in applications like communication systems and power electronics.
\end{itemize}

\end{document}